


************************************************
** Modulhandbuch
************************************************

Die  Studierenden  erläutern  die  Grunddimensionen  der  subjektiven Sinnesphysiologie  (Intensität,  Qualität,  Raum,  Zeit)  und  die  Begriffe Schwellenbestimmung, Weber-Quotient, two-alternative-forced-choice-Methode   sowie   die   psychophysischen   Gesetze   von   Weber/Fechner  und  Stevens.  Die  Studierenden  lernen  was  adäquate Reize sind und was unter Transduktion und Transmission, Frequenzkodierung und Generatorpotential, lateraler und rekurrenter Hemmung sowie  rezeptives  Feld,  zu  verstehen  ist.  Sie  erläutern  unspezifische und spezifische sensorische Bahnen.


************************************************
** Relevante Themen aus dem IMPP Gegenstandskatalog
************************************************

12.5 Signalverarbeitung im Nervensystem

    12.5.2 Verarbeitung in Neuronenpopulationen
        Vorwärtshemmung; Rückwärtshemmung; laterale Inhibition; Kontrastverschärfung, zeitliche Kodierung und Frequenzkodierung, Netzwerk-Oszillationen

12.6. Funktionsprinzipien sensorischer Systeme

    12.6.1. allgemeine Aspekte
        
        primäre und sekundäre Sinneszellen; Sinnesreize; Sinnesmodalitäten und Submodalitäten (Qualitäten); adäquater Reiz; Absolutschwelle; Intensitätsschwelle; Raumschwelle (Zweipunktschwelle); psychophysische Beziehungen
        
        Gesetz der spezifischen Sinnesenergien
        
    12.6.2 Rezeptorpotential/Sensorpotential
    
        Transduktion spezifischer Reize in elektrische Signale; mechanisch, thermisch und durch Liganden gesteuerte Ionenkanäle; Second-messenger-gesteuerte Ionenkanäle; graduierte Veränderungen der Ionenleitfähigkeit; adäquater Reiz; Kodierung der Reizintensität; Arbeitsbereich der Rezeptoren; Adaptation und Sensitisierung
        
        Missempfindungen; Parästhesien; Toxinwirkungen und Second-messenger-Prozesse; Schwellenveränderungen (z. B. zentrale oder periphere)
        
    12.6.3 Transformation der Reize

        elektrotonische Leitung der Rezeptorpotentiale; Generierung von Aktionspotentialen; Frequenzkodierung der Reizintensität; geschwindigkeitsabhängiges Rezeptorverhalten (proportional, differentiell); rezeptives Feld
        
        



************************************************
** Notizen Brandes Chapter 49
************************************************

49.1 Sinnesmodalitäten und Selektivität der Sinnesorgane für adäquate Reizformen
    49.1.1. Sinnesmodalitäten und Sinnesqualitäten
        Gesetz der spezifischen Sinnesenergien
            Definition Modalität, Qualität
        Qualitätsschwellen
        Einteilung der Sinne
    49.1.2 Adäquate Reize
        Adäquate und inadäquate Reize
            Meistens aber nicht immer optimal (d.h. minimale Energie), aber siehe z.B. Kaltsensoren und Menthol
        Ursachen der spezifischen Reizempfindlichkeit
   49.1.3 Sinnesorgane als Sensoren in Regelkreisen
        Regelkreise, die nicht auf bewusste Empfindung angewiesen sind
49.2 Informationsübertragung in Sensoren und afferenten Neuronen
    49.2.1 Transduktionsprozess
        Sensoren
        Transduktion
            Übersetzung eines Reizes in eine Membranpotenzialänderung
    49.2.2 Transduktion chemischer Reize
        Funktion von Chemosensoren
    49.2.3 Transduktion thermischer Reize
        Molekulare Strukturen von Thermosensoren
        Funktion und Arbeitsbereiche von Thermosensoren
    49.2.4 Transduktion mechanischer Reize
        Funktion von Mechanosensoren
        Arbeitsweise von Mechanosensoren
    49.2.5 Kodierung der Reizintensität
        Sensorschwelle und der Arbeitsbereich von Sensoren
        Empfindlichkeit des Transduktionsprozesses
            Verstärkung
    49.2.6 Prozess der Transformation
        Das Sensorpotenzial als Generatorpotenzial
        Umcodierung zu Aktionspotenzialen
            Amplitude zu Frequenz
49.3 Informationsverarbeitung im neuronalen Netz
    49.3.1 Periphere (primäre) und zentrale (sekundäre) rezeptive Felder
        Primäre rezeptive Felder
        Sekundäre rezeptive Felder und Funktionsanpassung
    49.3.2 Sensorische Bahnen als neuronale Netzwerke
        Allgemeine Struktur sensorischer Bahnen
            sensorische Bahn
        Divergenz und Konvergenz sensorischer Bahnen
        Redundanz sensorischer Bahnen
    49.3.3 Hemmende Synapsen im neuronalen Netz
        Funktion von hemmenden Synapsen
    49.3.4 Hemmende rezeptive Felder
        Laterale Hemmung
        Kontrastverschärfung
        Aufgaben der Eigenschaftsextraktion im neuronalen Netz
    49.3.5 Multisensorische Hirnregionen
            ARAS
49.4 Sinnesphysiologie und Wahrnehmungspsychologie
    49.4.1 Empfindungen und Wahrnehmungen
        Wahrnehmung als erfahrungsgeprägte Empfindung
        Wahrnehmung von Vexierbildern
            z.B. Agnosie
    49.4.2 Bindungsproblem
49.5 Sensorische Schwellen
    49.5.1 Entwicklung des Schwellenkonzeptes
        Reizschwelle
        Unterschiedsschwelle
            Weber-Gesetz
    49.5.2 Methoden der Messung von Sinnesschwellen
        Statische Betrachtung von Schwellen
        
            
    
        


************************************************
** Notizen VL Allgemeine Sinnesphysiologie
************************************************

Notizen VL Allgemeine Sinnesphysiologie, HMU (Stumpf) - Teil 1

1. Einführung
2. Einteilung der Sinnesrezeptoren
3. Vom Sensorpotential zur Aktionspotentialserie
4. Adaptation und dynamisches Antwortverhalten
5. Sinnesrezeptoren als Sensoren in Regelkreisen


1. Einführung: 

- Aufgaben der Allgemeinen Sinnesphysiolgie
- Geschichte des Hirn-Bewusstseins Problems
- Terminologie: Objektive Sinnesphysiolgie vs subjektive (Wahrnehmungs-) Sinnesphysiologie
    Reflex - Empfindung - Wahrnehmung, Reaktion, Bewusstsein
    Beispiel: Bär wird von Biene gestochen
- Hirn-Bewusstseins-Problem: Beispiel Roboter
- Einführung Sinnessysteme: 
    Wandeln physikalische und chemische Reize in neuronale Erregungen um und kodieren dabei Intensität, Qualität, Dauer. 
    Verarbeitung im ZNS erlaubt Erkennung relevanter Reizmuster, um angemessen zu reagieren. 
    Nicht-getreue Abbildung der Umwelt, sondern Kodierung von handlungsrelevanten Ereignissen. 
  
    
2. Einteilung der Sinnesrezeptoren

- Evolution von Sinnesrezeptoren: 
    Beispiel: mechanoensitiver Kanal zur Angleichung von Osmolarität
- Mechanorezeptoren beim Menschen: zwischen Adventitia und Media von großen Gefäßen: Mechanismus
- Thermorezeptoren
- Chemorezeptoren
- Photorezeptoren: Beispiel Auge
- Einteilung Sinnesrezeptoren: Primäre und sekundäre Sinneszellen

3. Vom Sensorpotential zur Aktionspotentialserie

- Begriffsklärung: Rezeptor, Sensor
    Rezeptor: Molekülkomplex, Sensor: Struktur oder Zelle auf der sich der Rezeptor befindet
    Beispiel: Barorezeptorreflex: Codierung von Amplitude als Frequenz

- Rezeption - Transduktion - Transformation - Konduktion. Beispiel: Pacini Körperchen der Haut. 

- Transduktionsmechanismen:
    Mechanorezeption: Mechanische Deformation -> Kanalprotein auf
    Geruchsrezeption: Second messenger Kaskade, Gs
    Photorezeptor: Schließen des Kanalproteins bei Licht, Gi
    Dauer, Ausmaß der Verstärkung. (Mehr Zwischenschritte: mehr Verstärkung, längere Dauer)
    
- Modulation von Transduktion und Transformation. Z.B. Rezeptorkanal für Säure oder Hitze, Entzündungsmediatoren

4. Adaptation und dynamisches Antwortverhalten

- Abbildung: Frequenz in Abhängigkeit von Zeit bei gleichbleibendem Reiz, Adaptation
- Tonischer Rezeptor vs phasischer Rezeptor, phasisch-tonische Rezeptoren, Nozizeptoren als Ausnahme (Antwort nimmt zu!)
- Die Mehrzahl aller "klassischen" Sinneszellen besitzt phasisch-tonische Eigenschaften, Ausnahme: Nozizeption.


5. Sinnesrezeptoren als Sensoren in Regelkreisen

- Wiederholung Reflexbogen, Reflexbogen als Regelkreis
- Beispiel für Reflexbogen: Patellarsehnenreflex
- Beispiel: Regulation des Blutdrucks

Zusammenfassung


Notizen VL Allgemeine Sinnesphysiologie, HMU (Stumpf) - Teil 2 [19:43]

 
1. Modalität, Qualität, Quantität
2. Adäquate und inadäquate Reize
3. Sensorische Schwellen
4. Psychophysische Beziehungen
5. Informationsverarbeitung in neuronalen Netzen

1. Modalität, Qualität, Quantität

- Was sind die Sinne? "Klassische 5 Sinne" + Lagesinn, Sensore für Blutruck, pO2, pCO2, pH, Glukose, etc., Schmerz
- Modalität, Qualität, Quantität
    Modalität: Was wird wahrgenommen? 
    Qualität: E.g. Farbe
    Quantität: Amplitude des Reizes
- Tabelle: Modalität, Qualität, Quantität

2. Adäquate und inadäquate Reize

- Adäquate vs inadäquate Reize:
    Adäquate Reize: Minimale Energie nötig, um Rezeptor zu erregen, z.B. Lichtquantum beim Photorezeptor
    Inadäquate Reize: Rezeptor wird bei höherer Einwirkung erregt, z.B. Faustschlag auf Auge, Sterne sehen 

- Gesetz der spezifischen Sinnesenergie:
    Nicht der Reiz, sondern das gereizte Sinnesorgan bestimmt die Wahrnehmung (Reizmodalität)
    
- Medizinisches Beispiel: Patient berichtet über Lichtblitze im Auge: Hinweis auf Ablösung der Netzhaut oder Tumor der Sehbahn, Epilepsie, Drogenkonsum, ...
    
- Zentrale Interpretation von Signalen aus der Peripherie hängt alleine davon ab, über welchen neuronalen Weg diese ins ZNS gelangen
    


3. Sensorische Schwellen

- Reizschwelle, Erkennungsschwelle, Unterschiedsschwelle, Sättigungsschwelle
- Ermittlung der Reizschwelle
    Konstantreizmethode (method of constant stimuli): #
        Kann etwas unterschiedliche Ergebnise beim selben Probenden gewinnen (je nach Serie)
        Psychometrische Funktion: Bruchteil der wahrgenommenen Reize als Funktion der Reizstärke
        Schwelle: Wo wurde die Hälfte aller Reize wahrgenommen?
    Grenzmethode (method of limits)
        z.B. Audiologie
        aufsteigendes, absteigendes Verfahren
        Absolutschwelle als Mittelwert der Grenzen nach mehreren Messungen
- Untersuchung von Reizschwellen ist auch an Tieren möglich
- Erkennungsschwelle: Niedrisgste Konzentration bei der eine Egenschaft erkannt wird/beschrieben werden kann
- Bsp: Farbtafel für Farbenblindheit.
- Sättigungsschwelle: Intensität oberhalb derer eine weitere Steigerung keinen stärkeren sensorischen Eindruck hervorruft
- Reiz, Empfindung, Wahrnehmung
    Empfindung skaliert nicht unbedingt linear mit dem Reiz, aber Reihenfolgen können etabliert werden (z.B. leicht, schwerer, noch schwerer)
    Vergleich der Empfindungsstärke setzt Wahrnehmung voraus
- Unterschiedsschwelle \(\phi\): kleinster wahrnehmbarer Unterschied in der physikalischen Reizstärke
    JND (Just Noticable Difference) \(\psi\): kleinster bemerkbarer Unterschied in der Empfindungsstärke
- Die Unterschiedsschwelle ändert sich mit der Reizstärke: Bsp leeres Glas vs volles Glas - wahrnehmbar. Aber zusätzliches Glas in einem Kübel wasser - nicht wahrnehmbar
- Webersches Gesetz: Die von einem Sinnesorgan gerade noch feststellbare Veränderung eines Rezies steht in der Regel in einem festen Verhältnis zu der absoluten Höhe des Reizes.
    Gerade noch feststellbare Veränderung / Höhe des Reizes = konstant
    
4. Psychophysische Beziehungen

- Weber-Fechner Gesetz: 
    Weber Gesetz: \(\frac{\Delta \phi}{\phi} = K\)    
    \(\phi \dots\) Unterschiedesschwelle, K konstant
    Fechner Gesetz: \(\Delta \psi = k\frac{\delta \phi}{\phi}\)
    \(\psi \dots JND\)
    
    Weber-Fechner Gesetz durch Integration: 
    
    \(\psi = k\log \left(\frac{\phi}{\phi_0}\right)\)

    \(\phi_0\): kleinste feststellbare Reizstärke

- Abbildung: Empfindungsstärke (JND) in Abhängigkeit von Reizstärke
    Viele klassische Sinnessysteme folgen Weber-Fechner
    
- Beispiel: Gehör
    Schalldruck (Pascal): Blätterrauschen, normale Unterhaltung, Disko, Gehörschaden, Schmerzschwelle, Düsenflugzeug in 30m Abstand
    Schalldruckpegel: \(L_p = 20 \log \frac{P}{P_0}\)
    \(P_0=20 \mu\text{P}\)

- Ausnahme zu Weber-Fechner: Schmerz
    Muskellänge: linear
    Potenzfunktion nach Stevens 
    \( \psi = k(\phi-\phi_0)^a\)
    \(a\) bestimmt ob log, linear oder exponentiell
    
- Empfindungsstärke muss in der Regel skaliert werden. Unterschiedliche Arten von Skalen. 

- Intermodaler Internsitätsvergleich. Bsp: Farbskala für Schmerzen

- Rasch adaptierende Mechanosensoren der Haut
    Mechanosensoren in großer Dichte an der Fingerbeere, weniger Dicht an der Handfläche
    Psychophysische Beziehung bei Erhöhung der Reizamplitude. Gemessen wird 1. Wahrnehmung der Probanden, 2 Elektrophysiologische Antwort.
    An der Fingerbeere sind beide nah beieinander, an der Handfläche klaffen elektrophysiologische Messung und Wahrnehmung der Probanden weit auseinander (Ephys sehr ähnlich aber Wahrnehmung unterschiedlich)

5. Informationsverarbeitung in neuronalen Netzen

- Primär rezeptive Felder
    sekundär rezeptive Felder von nachverschalteten Neuronen (Konvergenz, Divergenz)
- Laterale Hemmung als Konstrastverstärkung
- Beispiel laterale Hemmung: Einstich einer Nadel in die Haut. Beispiel: Tasten, Braille-Schrift
- Beispiel für Informationsverarbeitung in neuronalen Netzen: Retina
- Photorezeptor des Auges
- Laterale Hemmung in der Retina: On/off Bipolarzellen
- Beispiel: Optische Täuschung: Helligkeit vor Hintergrund 
- Exkurs: Andere optische Täauschungen (Netzwerk-Level)
- Zusammenfassung: Informationsverarbeitung in neuronalen Netzen; andere Sinnesmodalitäten können verstärkend oder hemmend eingreifen
- Zusammenfassung: objektive Sinnesphysiologie, Wahrnehmnungsphysiologie


********************************
** Fragen/Nachlesen:
********************************

- Liste von Reizmodalitäten
- Definition adäquater/inadäquater Reiz
- Farbenblindheitstest gutes Beispiel für Erkennungsschwelle?
- Weber-Fechner herleiten?


********************************
** Fragen beantwortet
********************************

- Unterschied Unterschiedsschwelle, JND? (Bzw. lässt sich JND überhaupt anders quantifizieren als durch die Unterschiedsschwelle?) - Ah OK, Weber-Fechner 


********************************
** Ideas/Connections
********************************

- Gesetz der spezifischen Sinnesenergien: Ausnahme Synästhesie
- Qualitätsschwellen: Beispiel Instrument stimmen, Beispiel: Shiny App Farbtest
- Frage: kleines vs großes rezeptives Feld, Fingerkuppe vs. Rumpf 
- Man who mistook his wife for a hat: glove als Beispiel für Agnosie
- Schwellenreiz-Methoden Beispiel: Inkscape-Punkte


Can take from ZJE Psychophysics lecture:
- Illustration of "naive"  view
- Examples of top-down processes, e.g. "the dress", gorilla experiment
- colour-ordering task? 
- psychophysics methods 
- Checker-shadow illusion
- Smell of cooking picture
