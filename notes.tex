


Notizen VL Allgemeine Sinnesphysiologie, HMU (Stumpf) - Teil 1

1. Einführung
2. Einteilung der Sinnesrezeptoren
3. Vom Sensorpotential zur Aktionspotentialserie
4. Adaptation und dynamisches Antwortverhalten
5. Sinnesrezeptoren als Sensoren in Regelkreisen


1. Einführung: 

- Aufgaben der Allgemeinen Sinnesphysiolgie
- Geschichte des Hirn-Bewusstseins Problems
- Terminologie: Objektive Sinnesphysiolgie vs subjektive (Wahrnehmungs-) Sinnesphysiologie
    Reflex - Empfindung - Wahrnehmung, Reaktion, Bewusstsein
    Beispiel: Bär wird von Biene gestochen
- Hirn-Bewusstseins-Problem: Beispiel Roboter
- Einführung Sinnessysteme: 
    Wandeln physikalische und chemische Reize in neuronale Erregungen um und kodieren dabei Intensität, Qualität, Dauer. 
    Verarbeitung im ZNS erlaubt Erkennung relevanter Reizmuster, um angemessen zu reagieren. 
    Nicht-getreue Abbildung der Umwelt, sondern Kodierung von handlungsrelevanten Ereignissen. 
  
    
2. Einteilung der Sinnesrezeptoren

- Evolution von Sinnesrezeptoren: 
    Beispiel: mechanoensitiver Kanal zur Angleichung von Osmolarität
- Mechanorezeptoren beim Menschen: zwischen Adventitia und Media von großen Gefäßen: Mechanismus
- Thermorezeptoren
- Chemorezeptoren
- Photorezeptoren: Beispiel Auge
- Einteilung Sinnesrezeptoren: Primäre und sekundäre Sinneszellen

3. Vom Sensorpotential zur Aktionspotentialserie

- Begriffsklärung: Rezeptor, Sensor
    Rezeptor: Molekülkomplex, Sensor: Struktur oder Zelle auf der sich der Rezeptor befindet
    Beispiel: Barorezeptorreflex: Codierung von Amplitude als Frequenz

- Rezeption - Transduktion - Transformation - Konduktion. Beispiel: Pacini Körperchen der Haut. 

- Transduktionsmechanismen:
    Mechanorezeption: Mechanische Deformation -> Kanalprotein auf
    Geruchsrezeption: Second messenger Kaskade, Gs
    Photorezeptor: Schließen des Kanalproteins bei Licht, Gi
    Dauer, Ausmaß der Verstärkung. (Mehr Zwischenschritte: mehr Verstärkung, längere Dauer)
    
- Modulation von Transduktion und Transformation. Z.B. Rezeptorkanal für Säure oder Hitze, Entzündungsmediatoren

4. Adaptation und dynamisches Antwortverhalten

- Abbildung: Frequenz in Abhängigkeit von Zeit bei gleichbleibendem Reiz, Adaptation
- Tonischer Rezeptor vs phasischer Rezeptor, phasisch-tonische Rezeptoren, Nozizeptoren als Ausnahme (Antwort nimmt zu!)
- Die Mehrzahl aller "klassischen" Sinneszellen besitzt phasisch-tonische Eigenschaften, Ausnahme: Nozizeption.


5. Sinnesrezeptoren als Sensoren in Regelkreisen

- Wiederholung Reflexbogen, Reflexbogen als Regelkreis
- Beispiel für Reflexbogen: Patellarsehnenreflex
- Beispiel: Regulation des Blutdrucks

Zusammenfassung


Notizen VL Allgemeine Sinnesphysiologie, HMU (Stumpf) - Teil 2 [19:43]

 
1. Modalität, Qualität, Quantität
2. Adäquate und inadäquate Reize
3. Sensorische Schwellen
4. Psychophysische Beziehungen
5. Informationsverarbeitung in neuronalen Netzen

1. Modalität, Qualität, Quantität

- Was sind die Sinne? "Klassische 5 Sinne" + Lagesinn, Sensore für Blutruck, pO2, pCO2, pH, Glukose, etc., Schmerz
- Modalität, Qualität, Quantität
    Modalität: Was wird wahrgenommen? 
    Qualität: E.g. Farbe
    Quantität: Amplitude des Reizes
- Tabelle: Modalität, Qualität, Quantität

2. Adäquate und inadäquate Reize

- Adäquate vs inadäquate Reize:
    Adäquate Reize: Minimale Energie nötig, um Rezeptor zu erregen, z.B. Lichtquantum beim Photorezeptor
    Inadäquate Reize: Rezepotr wird bei höherer Einwirkung erregt, z.B. Faustschlag auf Auge, Sterne sehen 

- Gesetz der spezifischen Sinnesenergie:
    Nicht der Reiz, sondern das gereizte Sinnesorgan bestimmt die Wahrnehmung (Reizmodalität)
    
- Medizinisches Beispiel: Patient berichtet über Lichtblitze im Auge: Hinweis auf Ablösung der Netzhaut oder Tumor der Sehbahn, Epilepsie, Drogenkonsum, ...
    
- Zentrale Interpretation von Signalen aus der Peripherie hängt alleine davon ab, über welchen neuronalen Weg diese ins ZNS gelangen
    


3. Sensorische Schwellen

- Reizschwelle, Erkennungsschwelle, Unterschiedsschwelle, Sättigungsschwelle
- Ermittlung der Reizschwelle
    Konstantreizmethode (method of constant stimuli): #
        Kann etwas unterschiedliche Ergebnise beim selben Probenden gewinnen (je nach Serie)
        Psychometrische Funktion: Bruchteil der wahrgenommenen Reize als Funktion der Reizstärke
        Schwelle: Wo wurde die Hälfte aller Reize wahrgenommen?
    Grenzmethode (method of limits)
        z.B. Audiologie
        aufsteigendes, absteigendes Verfahren
        Absolutschwelle als Mittelwert der Grenzen nach mehreren Messungen
- Untersuchung von Reizschwellen ist auch an Tieren möglich
- Erkennungsschwelle: Niedrisgste Konzentration bei der eine Egenschaft erkannt wird/beschrieben werden kann
- Bsp: Farbtafel für Farbenblindheit.
- Sättigungsschwelle: Intensität oberhalb derer eine weitere Steigerung keinen stärkeren sensorischen Eindruck hervorruft
- Reiz, Empfindung, Wahrnehmung
    Empfindung skaliert nicht unbedingt linear mit dem Reiz, aber Reihenfolgen können etabliert werden (z.B. leicht, schwerer, noch schwerer)
    Vergleich der Empfindungsstärke setzt Wahrnehmung voraus
- Unterschiedsschwelle: kleinster wahrnehmbarer Unterschied in der physikalischen Reizstärke
    JND (Just Noticable Difference): kleinster bemerkbarer Unterschied in der Empfindungsstärke
- Die Unterschiedsschwelle ändert sich mit der Reizstärke: Bsp leeres Glas vs volles Glas - wahrnehmbar. Aber zusätzliches Glas in einem Kübel wasser - nicht wahrnehmbar
- Webersches Gesetz: Die von einem Sinnesorgan gerade noch feststellbare Veränderung eines Rezies steht in der Regel in einem festen Verhältnis zu der absoluten Höhe des Reizes.





********************************
** Fragen/Nachlesen:
********************************

- Liste von Reizmodalitäten
- Definition adäquater/inadäquater Reiz
- Farbenblindheitstest gutes Beispiel für Erkennungsschwelle?
- Unterschied Unterschiedsschwelle, JND? (Bzw. lässt sich JND überhaupt anders quantifizieren als durch die Unterschiedsschwelle?) 